\documentclass[a4paper,12pt]{letter}
\usepackage{fancyhdr,color,graphicx,afterpage,enumitem}
\usepackage[top=50mm,head=40mm,bottom=40mm,foot=30mm,left=20mm,right=20mm]{geometry}
\usepackage{tabularx}
\usepackage{sansmathfonts}
\usepackage[scaled]{helvet}
\renewcommand\familydefault{\sfdefault} 
\usepackage[T1]{fontenc}
\renewcommand{\headrule}{
\vskip-13ex \hrule width\headwidth height\headrulewidth \vskip-\headrulewidth}
%== adjust line position -13.ex by trial an error, if # of lines in 
%== the header boxes below is changed 
\rhead{
 \begin{minipage}[t]{.4\textwidth}
  \raggedleft
   \includegraphics[width=.5\textwidth]{unsw}\vspace*{1ex}\ \\
   \begin{footnotesize}
   \begin{sf}
   \begin{tabular}{l@{\ :\ }l}
    T& +61 2 9065 4259\\
    E& B.Laws@unsw.edu.au\\
    \multicolumn{2}{l}{research.unsw.edu.au/people/dr-ben-laws}
   \end{tabular}
  \end{sf}
  \end{footnotesize}
 \end{minipage}
}
\lhead{
 \begin{minipage}[t]{.6\textwidth}
  \begin{footnotesize}
  \begin{sf}
  \raggedright
  {\textsc\textbf{UNSW FACULTY OF SCIENCE}}
  \vspace*{2ex}\ \\
          \textbf{Dr Benjamin A. Laws}\\
          Postdoctoral Fellow\\
          Molecular Photonics Laboratories\\
          School of Chemistry\\
          The University of New South Wales\\
          Sydney NSW 2052 Australia\\
  \end{sf}
  \end{footnotesize}
 \end{minipage}
}
\cfoot{}
\lfoot{}
\rfoot{}
\setlength{\parskip}{1em}
\pagestyle{plain}
% =============== Add your input below =================================
\address{\vspace*{-8ex}\ \\
} 
\signature{\sf \vspace*{-8ex}Dr Benjamin A. Laws.}
\begin{document}
\begin{sf}
%\date{ }                  % comment this line to specify today's date
\begin{letter}{%
Editor\\
The Journal of Chemical Physics
}
\opening{Dear Editor,}
\thispagestyle{fancy}

The following manuscript is for your consideration for publication in
The Journal of Chemical Physics:

\begin{tabularx}{.95\textwidth}{ll}
Title: & \hspace*{.4em}\mbox{\it Velocity Map Imaging Spectroscopy of C$_2$H$^-$ and C$_2$D$^-$:}\\
 & \hspace*{.4em}\mbox{\it a benchmark study of vibronic coupling interactions}
\\
 \\
Authors: & 
{\begin{tabular}[t]{lll}
Dr Benjamin~A.~Laws & Postdoctoral Fellow & B.Laws@unsw.edu.au\\
Zachariah~D.~Levey & PhD Student & Z.Levey@unsw.edu.au\\
Prof. Andrei Sanov & Professor & Sanov@arizona.edu\\
Prof. John F. Stanton & Professor & Johnstanton@chem.ufl.edu\\
Prof. Timothy~W.~Schmidt & Professor & Timothy.Schmidt@unsw.edu.au\\
Dr Stephen~T.~Gibson & Senior Fellow & Stephen.Gibson@anu.edu.au
\end{tabular}}\\
 \\
\parbox[t]{6em}{
	Addresses: \\ 
	(1,2,5)} & 
{\begin{tabular}[t]{l}
		School of Chemistry, The University of New South Wales\\
		Sydney, NSW 2052, Australia
\end{tabular}} \\
(1, 6) & 
{\begin{tabular}[t]{l}
Research School of Physics and Engineering, The Australian National University\\
Canberra, ACT 2601, Australia
\end{tabular}} \\
(3) & 
{\begin{tabular}[t]{l}
		Department of Chemistry \& Biochemistry, The University of Arizona\\
		Tucson, Arizona 85721, United States
\end{tabular}} \\
(4) & 
{\begin{tabular}[t]{l}
		Department of Chemistry, University of Florida\\
		Gainesville, Florida 32611, United States
\end{tabular}} \\


\\
\parbox[t]{6em}{
Conflict of Interest} &
\hspace*{.4em}The authors declare no competing financial interest.
\end{tabularx}

\vspace*{0ex}\ \\
\textbf{Significance}\\[.5ex]
This research uses High Resolution Photoelectron Imaging (HR-PEI) to examine complex vibronic coupling effects in the ethynyl radical C$_2$H. Monohydride carbon chains play an important role in combustion and interstellar chemistry, with C$_2$H reported to be one of the most abundant molecules in the universe. However, these radicals are exceedingly difficult to model, due to the introduction of vibronic coupling interactions between the close lying $\tilde{X}^2\Sigma^+$ and $\tilde{A}^2\Pi$ electronic states. As our knowledge of interstellar chemistry relies critically on theory, a new approach is required that can both accurately model the complexity of the coupled spectra, and be efficiently extended to larger C$_{2n}$H radicals.

In this paper we apply a quasidiabtaic approach to construct a QVC (quadratic vibronic coupling) model Hamiltonian to account for the Renner-Teller and pseudo Jahn-Teller coupling in C$_2$H. We show how the Hamiltonian can be parameterised using the outputs of standard quantum chemical \emph{ab-initio} calculations, to simulate the vibronic coupling effects in the photoelectron spectrum of C$_2$H and C$_2$D.  Comparisons to the HR-PEI experimental spectra confirm that the QVC model accurately reproduces the coupling effects observed near the $\tilde{A}^2\Pi$ state origin. Anion HR-PEI spectroscopy maps both of the $\tilde{X}^2\Sigma^+$ and $\tilde{A}^2\Pi$ surfaces on an equal footing from the anion $\tilde{X}^1\Sigma^+$ state, allowing for a direct comparison with the \emph{ab-initio} modelling. We also show that detailed analysis of the experimental electron anisotropy allows for transitions with $\sigma$ and $\pi$ vibronic symmetry to be distinguished, providing definitive assignments of the calculated transitions.


As the QVC quasidiabatic model can be readily extended to larger systems, this work provides a roadmap for future studies to follow, which will help to determine the role of C$_{2n}$H and C$_{2n}$H$^-$ in interstellar chemistry.

\vspace*{0ex}\ \\ 
\textbf{Recommended Reviewers}\\[.5ex]
\begin{tabularx}{.5\textwidth}{lll}
	Caroline~Chick~Jarrold & Indiana U. & cjarrold@indiana.edu\\
	Anna Krylov  & USC  & krylov@usc.edu\\
	Richard Mabbs  & Washington U.   & mabbs@wustl.edu\\
	Ryan Fortenberry & U. Mississippi & r410@olemiss.edu\\
	Mark Johnson & Yale & mark.johnson@yale.edu\\
	Michael~Heaven & Emory & mheaven@emory.edu\\
	Arthur~Suits & U. Missouri & suitsa@missouri.edu\\
\end{tabularx}

\vspace*{0ex}\ \\ 
\textbf{Statement}\\[.5ex]
This manuscript is not being considered by any other journal.

%\closing{Regards,\\
\   
\includegraphics[width=.2\textwidth,clip]{signatureXX}
               \closing{Regards,}
\end{letter}
\end{sf}
\end{document}
